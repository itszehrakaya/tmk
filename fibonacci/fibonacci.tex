\documentclass{article}
\usepackage{graphicx} % Required for inserting images
\usepackage{amsmath}
\usepackage{amssymb}
\usepackage{amsthm}
\usepackage{xcolor,lipsum}

\usepackage{librebaskerville}

% Defining custom colors
\definecolor{bgRGB}{RGB}{255,255,212} 
\definecolor{mypink2}{RGB}{95, 51, 41}



\pagecolor{bgRGB}

\makeatletter
\newcommand{\globalcolor}[1]{%
  \color{#1}\global\let\default@color\current@color
}
\makeatother

\AtBeginDocument{\globalcolor{mypink2}}



\title{Fibonacci Dizileri}
\author{Zehra Kaya \\ Melih Mert Oskay}
\begin{document}


\maketitle
% \tableofcontents



\section*{Problemler}

%\subsection*{Fibonacci Dizisindeki İlginç Özdeşlikler }

%commment empty line 
% Remove the double backslashes

Fibonacci dizisi, ilk ve ikinci terimin her ikisinin de 1’e eşit olduğu ve her bir sonraki terimin kendisinden önceki iki terimin toplamı olduğu bir tam sayı dizisidir.
\begin{align*}
F_0 &= 1\\
F_1 &= 1\\
F_n &= F_{n-1} + F_{n-2} \quad , \quad n \geq 2
\end{align*}

Fibonacci dizisinin ilginç bir özelliği ise, ilk n terimin toplamının, n+2. terimin 1 eksiğine eşit olmasıdır:\\
\begin{center}
    $F_0 + F_1 + \dots F_n = F_{n+2} -1 $\\
\end{center}


    

Bu özdeşliğin doğru olduğunu gösterin. ( Peki bu özdeşlik ilk iki terimi 1 olmayan fibonacci dizileri için de geçerli midir? )


\end{document}